\section{Disposições Gerais}\label{sec:rules}

\subsection{Obrigações do \textbf{Locatário}}\label{sub-seb:renter-rules}

\begin{enumerate}
    \item Destinar o imóvel (\tipoImovel) somente para fim \FinalidadeImovel, não podendo ser mudado sua finalidade;
    \item Além do pagamento do aluguel, a satisfazer o pagamento do consumo de água, luz, esgoto e IPTU, bem como todos os demais tributos municipais que recaiam sobre o imóvel locado;
    \item Sempre que notificado, apresentar ao \textbf{Locador} a comprovação dos pagamentos junto ao próprio;
    \item Manter o objeto da locação no mais perfeito estado de conservação e limpeza, para assim restituí-lo ao \textbf{Locador}, quando finda ou rescindida a ligação, correndo por sua conta exclusivos as despesas necessárias para esse fim, notadamente, as que se referem à conservação de pinturas, portas comuns, fechaduras, trincos, puxadores, vitrais e vidraças, lustres, instalações elétricas, torneiras, aparelhos sanitários e quaisquer outras, inclusive obrigando-se a pintá-lo novamente em sua desocupação, com tintas e cores iguais às existentes; 
    \item Não transferir este contrato, não sublocar, não ceder ou emprestar, sob qualquer pretexto e de igual forma alterar a destinação da locação, não constituindo o decurso do tempo, por si só, na demora das locadoras em reprimir a infração, assentimento à mesma;
    \item Encaminhar ao \textbf{Locador} todas as notificações, avisos ou intimações dos poderes públicos que foram entregues no imóvel, sob pena de responder pelas multas, correção monetária e penalidades decorrentes do atraso no pagamento ou satisfação no cumprimento de determinações por aqueles poderes;
    \item No caso de qualquer obra, reforma ou adaptação, devidamente autorizado pelo \textbf{Locador}, repor por ocasião da entrega efetiva das chaves do imóvel locado, ficará incorporado ao imóvel, não podendo exigir qualquer indenização ou ressarcimento;
    \item Facultar ao \textbf{Locador} ou ao seu representante legal examinar ou vistoriar o imóvel sempre que for para tanto solicitado, bem como no caso do imóvel ser colocado à venda, permitir que interessados o visitem, devendo, para tanto, fixar o respectivo horário, para que se realizem as visitas; 
    \item Na entrega do imóvel, verificando-se a infração pelo \textbf{Locatário} de quaisquer das cláusulas que se compõe este contrato, e que o prédio necessite de algum conserto ou reparo, ficará o mesmo \textbf{Locatário} pagando o aluguel até a entrega das chaves;
    \item Findo o prazo deste contrato, por ocasião da entrega das chaves, ao \textbf{Locador} mandará fazer uma vistoria no prédio locado, a fim de verificar se o mesmo se acha nas condições em que fora recebido pelo \textbf{Locatário};
\end{enumerate}

\subsection{Outras Disposições}\label{sub-seb:general-rules}

\begin{enumerate}
    \item Fica desde já estabelecido que, caso o governo venha a tomar medidas sobre a periodicidade do reajuste, neste contrato de locação automaticamente prevalecerá o menor prazo para ser reajustado, sempre pelo índice a ser determinado pelos órgãos competentes.
    \item Ocorrendo atraso no pagamento do aluguel será o mesmo acrescido de multa de 10\% (dez por cento), além de juros moratórios de 1\% (um por cento) ao mês e correção monetária entre a data d vencimento até o efetivo pagamento.
    \item A falta de pagamento do aluguel e encargos dentro do prazo estipulado constituirá por si só em mora o \textbf{Locatário} independente de qualquer aviso, notificação ou interpelação.
    \item A infração das obrigações, sem prejuízo de qualquer outra prevista em lei, por parte do \textbf{Locatário}, é considerada como de natureza grave, acarretando a rescisão contratual, com o consequente despejo e obrigatoriedade de imediata satisfação dos consectários contratuais e legais;
    \item Caso o objeto da locação vier a ser desapropriado pelos Poderes Públicos, ficará o presente contrato, bem como o \textbf{Locador}, exonerado de todas e quaisquer responsabilidades decorrentes. Ocorrerá a rescisão deste contrato de pleno direito no caso de desapropriação, incêndio ou acidente que sujeite o imóvel locado às obras que importem na sua reconstrução total, ou que impeçam o uso do mesmo por mais de trinta dias;
    \item Obriga-se ao \textbf{Locatário} a renovar expressamente novo contrato, caso venha a permanecer no imóvel. O novo aluguel, após o vencimento, será efetuado por convenção das partes.
    \item Toda e qualquer benfeitoria autorizada pelo \textbf{Locador}, ainda que útil ou necessária, ficará automaticamente incorporada ao imóvel, não podendo ao \textbf{Locatário} pretender qualquer indenização ou ressarcimento, bem como argüir direito de retenção pelas mesmas;
    \item A locação estará sempre sujeita ao Regime do Código Civil Brasileiro e à Lei nº 8.245/91, ficando assegurado ao \textbf{Locador} todos os direitos e vantagens conferidas pela legislação que vier a ser promulgada durante a locação;
    \item Findo o prazo deste Contrato, mas prorrogada a locação, por vontade das partes ou por disposição de Lei, todas as cláusulas ora estipuladas continuarão em pleno vigor e reguladoras das relações entre os contratantes, por prazo indeterminado até o final e efetiva restituição do imóvel locado;
    \item Fica convencionado que ao \textbf{Locatário} deverá fazer o pagamento dos aluguéis mensais pontualmente até o dia \diadepagamento de cada mês seguinte ao vencido, ficando esclarecido que, passado este prazo estará em mora, sujeito às penas impostas neste contrato. Após o dia \diadepagamento do mês seguinte ao vencido, o \textbf{Locador} poderá enviar o(s) recibo(s) de aluguéis e encargos da locação para cobrança através de advogado, mesmo que a cobrança seja realizada extra-judicialmente; no caso de cobrança judicial, pagará ao \textbf{Locatário} também as custas decorrentes;
\end{enumerate}

\subsection{Venda do Imóvel Locado}

Respeitado o direito de preferência do \textbf{Locatário}, caso o imóvel seja colocado à venda, o \textbf{Locatário} permitirá que os interessados na compra o visitem em dia e hora previamente indicados pelo \textbf{Locador} com ciência por escrito do \textbf{Locatário}.

\subsection{Foro}

Para todas as questões resultantes deste contrato, será competente o foro da situção do imóvel seja qual for o domicílio dos contratantes. Tudo quanto for devido em razão deste contrato e que não comporte processo executivo será cobrado em ação competente, ficando a cargo do devedor em qualquer caso, os honorários do advogado no importe de 20\%.

E assim, por estarem justos e contratados, assinam o presente contrato de locação em 02 (duas) vias de igual teor, e forma, na presença de duas testemunhas para fins de Direito.